\documentclass{article}

\setlength{\parskip}{1em}
\usepackage{indentfirst}

\usepackage[colorlinks=true, linkcolor=blue, urlcolor=blue, citecolor=blue]{hyperref}
\usepackage{amsmath,amsfonts,amssymb,amsthm,natbib,graphicx,mathtools}
\usepackage{braket}
\usepackage{enumitem}
\usepackage{hyperref,pgfplots,siunitx}
\usepackage[outdir=./]{epstopdf}

\sisetup{detect-all}
\pgfplotsset{compat=1.18}

\theoremstyle{definition}
\newtheorem{defn}{Definition}[section]
\newtheorem{rmk}[defn]{Remark}
\newtheorem{note}[defn]{Note}
\newtheorem{notation}[defn]{Notation}

\theoremstyle{plain}
\newtheorem{lemma}[defn]{Lemma}
\newtheorem{prop}[defn]{Proposition}
\newtheorem{thm}[defn]{Theorem}

\begin{document}

\begin{titlepage}
    \centering
    {\huge \bfseries Local Gauge Anomalies and The Standard Model \par}
    \vspace{1cm}
    {\Large \itshape E. Arnold \& K. Okeyre\par}
    \vfill
    {\Large October 2025 \par}
    {\Large Undergraduate Summer Research Project \par}
    {\Large Durham University \par}
\end{titlepage}

\section*{General structure}

    \begin{enumerate}
    \item { Generalise term 1 MathPhys to Quantum field theory. Explain symmetries (global syms and the different local syms) and their conserved quantities }
    \item { Explain what anomolies are and why gauge anomolies must be cancelled out }
    \item { Maybe talk about how they were cancelled out before Dai-Freed }
    \item { Explain Partition Functions }
    \item {Explain the operator formulation of QFT}
    \item {Explain the Path integral formulation}
    \item Try to explain gauge transformations and gauge vs global symmetries
    \item 
    \end{enumerate}

\section*{New General structure}

    \begin{enumerate}
    \item Explain the operator formulation of QFT
    \item Explain the Path integral formulation
    \item Try to explain gauge transformations and gauge vs global symmetries
    \item Explain the partition function and anomalies
    \end{enumerate}


\newpage

\section*{Abstract}
Abstract here
\newpage

\tableofcontents
\newpage

\section{Introduction}

The following report is aimed at undergraduate 
students familiar with lagrangian mechanics, hamiltonian mechanics,
quantum mechanics, and topology. For a guide to the expected prerequisite knowledge see \cite{mathphys1, mathphys2, topnotes}.

[The aim of this report is to offer a gentle 
introduction to quantum field theory, and offer 
insight into one example of where topology shows up 
in theoretical physics.] - CHANGE THIS

Explain structure of report here. 

\section{Quantum States}

One should be familiar with the notion of a quantum wavefunction, such as 
$\psi(x)$, which describes the position of a particle. 
However, in quantum field theory we deal with a more 
abstract concept called a quantum state.

\begin{defn}
Define Hilbert space
\end{defn}

\begin{notation}
We use the Bra-Ket notation for vectors in Hilbert space. A 
vector is written as $| \psi \rangle$ and a covector is 
written as $\langle \phi |$. We define \[ \langle \phi | \psi \rangle := \langle \phi | (| \psi \rangle) \]
Given a vector 
$| \psi \rangle$ the corresponding hermition transpose of 
this vector is $\langle \psi |$, meaning 
$|\psi|^2 = \langle \psi | \psi \rangle$.
\end{notation}

A quantum state is a mathematical entity that embodies all 
the known information of some given quantum system. There 
are two types of quantum states, pure and mixed. We will 
only explain pure quantum states in this section.

A (pure) quantum state is an abstract vector in a complex 
Hilbert space, denoted by $| \psi \rangle $. This Hilbert 
space is called our state space and often has infinite 
dimensions.

Observables of our quantum system correspond to Hermition 
operators on our state space. The eigenvalues of such an 
operator are real and correspond to possible observed 
values. It can be shown that the eigenstates of any 
Hermition operator form a basis for our state space. This 
allows us to relate quantum states to the more familiar 
notion of a quantum wavefunction.

Let $\langle \psi |$ be a quantum state and $\hat{X}$ be the position operator with eigenstates $|x \rangle$. Then $\psi (x) = \langle x | \psi \rangle$. WHY???


\section{The Heisenburg Picture of Quantum Mechanics}

One is usually introduced to quantum mechanics in the 
Schrödinger picture, however, in quantum field theory, 
it is more natural to consider the Heisenburg picture.

In the Schrödinger picture, the quantum states vary with 
time. For example, our quantum state may be represented 
by a wave function $\psi(x,t)$, which evolves in time. 
Conversely,
the operators that act on our state space, such as 
momentum, position, etc, are usually fixed with respect to time. 
The only exception to this is that the Hamiltonian may include 
a potential energy term that varies with time. 

Consider a time dependent quantum state 
$| \psi(t) \rangle$. This state evolves in time 
according the  Schrodinger equation. One can 
represent this via a unitary time-evolution operator 
$U(t,t_0)$ 
as follows;
\[ | \psi(t) \rangle = U(t,t_0) | \psi(t_0) \rangle .\]

In the Schrödinger 
picture, the expectation of a potentially time-dependent hermition operator, $A(t)$, in the 
state $| \psi(t) \rangle$,  can be written as 
\[\langle \hat{A}(t) \rangle = \langle \psi(t) | \hat{A}(t) | \psi(t) \rangle.\]
Choosing some reference time $t_0$ we can rewrite this as
\[\langle \psi(t_0) | \hat{U}\textsuperscript{\textdagger}(t,t_0) \hat{A}  (t) \hat{U}(t,t_0) | \psi(t_0) \rangle.\]
If we define 
\[\hat{A_H}(t) = \hat{U}\textsuperscript{\textdagger}(t,t_0) \hat{A}  (t) \hat{U}(t,t_0) \]
then we have 
\[\langle \hat{A}(t) \rangle = \langle \psi(t_0) | \hat{A_H}(t) | \psi(t_0) \rangle.\]

So we notice that we could have obtained this same 
expectation by considering a constant quantum state 
$| \psi(t_0) \rangle$ and a new operator that evolves with 
time. This is the idea behind the Heisenberg picture of 
quantum mechanics. One interprets the quantum states as 
being fixed in time, and it is the operators that evolve 
in time. 

In the Schrödinger picture, the Schrödinger equation 
governs the time evolution of quantum states. Hence, a 
natural question is, in the Heisenberg picture, how do the 
operators evolve in time?

One can derive the following equation
\[ \frac{d}{dt} \hat{A}_H(t) = \frac{1}{i \hbar} [\hat{A}_H(t), \hat{H}_H(t)] + \left[\frac{d}{dt} \hat{A}_S(t) \right]_H \]
where the subscripts $S$ and $H$ denote whether we are considering the element in the Schrodinger or Heisenberg picture respectively. 

\section{Classical field theory \& Noether's theorem}
Classical field theory is a generalisation of classical 
mechanics where instead of considering a collection of 
particles $\{q_i(t)\}_i$, we consider a function on 
spacetime, $\phi(\vec{x},t)$, called a field. The $i$ has 
been generalised to $\vec{x}$.

The Lagrangian density at a point is a function of 
fields and their derivatives, 
\[\mathcal{L} (\vec{x},t) = \mathcal{L}( \phi_a (\vec{x},t),\partial_{\mu} \phi_a (\vec{x},t) ),\]
where the subscript $a$ is because the density may depend on multiple fields.
The Lagrangian is then written as an integral of this Lagrangian density over space,
\[ L(t) = \int d^3\vec{x} \mathcal{L} (\vec{x},t) .\]
Our action is then written as a time integral of this Lagrangian,
\[ S = \int dt L(t) \]
or equivalently
\[ S = \int d^4x \mathcal{L} ( \phi_a,\partial_{\mu} \phi_a ). \]
From this, we can derive the new Euler-Lagrange equations,
\[ \frac{\partial \mathcal{L}}{\partial \phi_a} - \partial_{\mu} \left[ \frac{\partial \mathcal{L}}{ \partial (\partial_{\mu} \phi_a)} \right] =0 .\]

***NOT SURE IF THIS WILL BE NEEDED

The fields in the Lagrangian density are relativistic, 
meaning under Lorentz tranformations
\[ x'^{\mu} = \Lambda^{\mu}_{\nu} x^{\nu} \]
they transform as 
\[ \phi_i'(x') = R_l^j \phi_i(x) \]

**what is i? Does it matter?

We use the Minkowski metric 
\[ \eta_{\mu \nu} = diag(-1,1,1,1) \]
in these notes.

***DOWN TO HERE 

Define a symmetry

State Noether's thm



\section{Path Integral \& Generating Functional}
Given a particle position $x$ at time $t$ the probability amplitude of
a measurement at time $t'$ observing a position $x'$ is

\begin{align*}
  \begin{split}
  M(x', t'; x, t) &= \prescript{}{H}{\braket{x',t'|x, t}_{H}}
  \end{split} \\
  \begin{split}
  &= \braket{x'|\exp{\left[\frac{-i}{\hbar}\hat{H}(t-t')\right]}|x}
  \end{split}
\end{align*}\label{eq:1}
now we note that if the states are normalised
\begin{equation}
  \int{dy |y,t\rangle{} \langle{y,t|} } = \mathbb{I}
\end{equation}
we can write 
  \begin{equation}
    M(x', t'; x, t) = \sum_{n}{\psi_{n}(x')\psi_{n}^*(x)\exp{\left[\frac{-i}{\hbar}\hat{H}(t-t')\right]}}
  \end{equation}
which acts as our propagator
\begin{equation}
 \psi(x', t') = \int{dx \ M(x', t'; x, t) \psi(x, t)}
\end{equation}


\section{Scalar Fields \& Green's Functions}
\section{Wess-Zumino Consistency \& Ward Identity}
\section{Stora-Zumino Descent}
\section{Dai-Freed \& Index Theorems}

\newpage
\bibliographystyle{unsrt}
\bibliography{references}
\addcontentsline{toc}{section}{References}

\end{document}
