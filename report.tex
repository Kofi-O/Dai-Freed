\documentclass[11pt, a4paper]{article}

\usepackage{jheppub}

\setlength{\parskip}{1em}
\usepackage{indentfirst}

\usepackage[colorlinks=true, linkcolor=blue, urlcolor=blue, citecolor=blue]{hyperref}
\usepackage{amsmath,amsfonts,amssymb,amsthm,natbib,graphicx,mathtools}
\usepackage{braket}
\usepackage{enumitem}
\usepackage{hyperref,pgfplots,siunitx}
\usepackage[outdir=./]{epstopdf}
\usepackage{slashed}


\sisetup{detect-all}
\pgfplotsset{compat=1.18}

\theoremstyle{definition}
\newtheorem{defn}{Definition}[section]
\newtheorem{rmk}[defn]{Remark}
%\newtheorem{note}[defn]{Note}
\newtheorem{notation}[defn]{Notation}

\theoremstyle{plain}
\newtheorem{lemma}[defn]{Lemma}
\newtheorem{prop}[defn]{Proposition}
\newtheorem{thm}[defn]{Theorem}

\begin{document}

\begin{titlepage}
    \centering
    {\huge \bfseries Local Gauge Anomalies and The Standard Model \par}
    \vspace{1cm}
    {\Large \itshape E. Arnold \& K. Okeyre\par}
    \vfill
    {\Large October 2025 \par}
    {\Large Undergraduate Summer Research Project \par}
    {\Large Durham University \par}
\end{titlepage}

\section*{Abstract}
Abstract here
\newpage

\tableofcontents
\newpage

\section{Introduction}

The following report is aimed at undergraduate 
students familiar with lagrangian mechanics, hamiltonian mechanics,
quantum mechanics, and topology. For a guide to the expected prerequisite knowledge see \cite{mathphys1, mathphys2, topnotes}.

[The aim of this report is to offer a gentle 
introduction to quantum field theory, and offer 
insight into one example of where topology shows up 
in theoretical physics.] - CHANGE THIS

Explain structure of report here. 

\section{Quantum States}

One should be familiar with the notion of a quantum wavefunction, such as 
$\psi(x)$, which describes the position of a particle. 
However, in quantum field theory we deal with a more 
abstract concept called a quantum state.

\begin{defn}
Define Hilbert space
\end{defn}

\begin{notation}
We use the Bra-Ket notation for vectors in Hilbert space. A 
vector is written as $| \psi \rangle$ and a covector is 
written as $\langle \phi |$. We define \[ \langle \phi | \psi \rangle := \langle \phi | (| \psi \rangle) \]
Given a vector 
$| \psi \rangle$ the corresponding hermition transpose of 
this vector is $\langle \psi |$, meaning 
$|\psi|^2 = \langle \psi | \psi \rangle$.
\end{notation}

A quantum state is a mathematical entity that embodies all 
the known information of some given quantum system. There 
are two types of quantum states, pure and mixed. We will 
only explain pure quantum states in this section.

A (pure) quantum state is an abstract vector in a complex 
Hilbert space, denoted by $| \psi \rangle $. This Hilbert 
space is called our state space and often has infinite 
dimensions.

Observables of our quantum system correspond to Hermition 
operators on our state space. The eigenvalues of such an 
operator are real and correspond to possible observed 
values. It can be shown that the eigenstates of any 
Hermition operator form a basis for our state space. This 
allows us to relate quantum states to the more familiar 
notion of a quantum wavefunction.

Let $\langle \psi |$ be a quantum state and $\hat{X}$ be the position operator with eigenstates $|x \rangle$. Then $\psi (x) = \langle x | \psi \rangle$. WHY???


\section{The Heisenburg Picture of Quantum Mechanics}

One is usually introduced to quantum mechanics in the 
Schrödinger picture, however, in quantum field theory, 
it is more natural to consider the Heisenburg picture.

In the Schrödinger picture, the quantum states vary with 
time. For example, our quantum state may be represented 
by a wave function $\psi(x,t)$, which evolves in time. 
Conversely,
the operators that act on our state space, such as 
momentum, position, etc, are usually fixed with respect to time. 
The only exception to this is that the Hamiltonian may include 
a potential energy term that varies with time. 

Consider a time dependent quantum state 
$| \psi(t) \rangle$. This state evolves in time 
according the  Schrodinger equation. One can 
represent this via a unitary time-evolution operator 
$U(t,t_0)$ 
as follows;
\[ | \psi(t) \rangle = U(t,t_0) | \psi(t_0) \rangle .\]

In the Schrödinger 
picture, the expectation of a potentially time-dependent hermition operator, $A(t)$, in the 
state $| \psi(t) \rangle$,  can be written as 
\[\langle \hat{A}(t) \rangle = \langle \psi(t) | \hat{A}(t) | \psi(t) \rangle.\]
Choosing some reference time $t_0$ we can rewrite this as
\[\langle \psi(t_0) | \hat{U}\textsuperscript{\textdagger}(t,t_0) \hat{A}  (t) \hat{U}(t,t_0) | \psi(t_0) \rangle.\]
If we define 
\[\hat{A_H}(t) = \hat{U}\textsuperscript{\textdagger}(t,t_0) \hat{A}  (t) \hat{U}(t,t_0) \]
then we have 
\[\langle \hat{A}(t) \rangle = \langle \psi(t_0) | \hat{A_H}(t) | \psi(t_0) \rangle.\]

So we notice that we could have obtained this same 
expectation by considering a constant quantum state 
$| \psi(t_0) \rangle$ and a new operator that evolves with 
time. This is the idea behind the Heisenberg picture of 
quantum mechanics. One interprets the quantum states as 
being fixed in time, and it is the operators that evolve 
in time. 

In the Schrödinger picture, the Schrödinger equation 
governs the time evolution of quantum states. Hence, a 
natural question is, in the Heisenberg picture, how do the 
operators evolve in time?

One can derive the following equation
\[ \frac{d}{dt} \hat{A}_H(t) = \frac{1}{i \hbar} [\hat{A}_H(t), \hat{H}_H(t)] + \left[\frac{d}{dt} \hat{A}_S(t) \right]_H \]
where the subscripts $S$ and $H$ denote whether we are considering the element in the Schrodinger or Heisenberg picture respectively. 

\section{Classical field theory \& Noether's theorem}

\subsection{Classical field theory}
In classical mechanics we consider a countable set of particles
each with finitely many degrees of freedom
and generalised coordinates $q_i(t)$. These generalised coordinates ${\{ q_i(t) \}}_i$ specify
the system's configuration (position in configuration space) and together with the 
generalised conjugate momenta
${\left\{p_i(t) \coloneq \frac{\partial L}{\partial \dot{q}_i} \right\}}_i $ 
define the system's position in phase space.

In classical field theory we generalise this notion of configuration space to a
continuum with infinite degrees of freedom. The scalar field $\phi(t)$ can be seen as
the generalised coordinates of a continuum 
$\left( q_i(t) \xrightarrow{i \rightarrow \vec{x}} \phi(\vec{x}, t) \right)$ and for a system
of continua the set of scalar fields ${\{ \phi_k(\vec{x}, t) \}}_k$
specifies the system's configuration.


Subsequently, we generalise the classical Lagrangian $L(q_i(t), \dot{q}_i(t), t)$ via
\begin{equation}
  L(t) = \int{d^{3}\vec{x}\,\mathcal{L}(\phi_k(\vec{x}, t),\partial_{\mu}\phi_k(\vec{x}, t), \vec{x}, t)}
\end{equation}
where $\mathcal{L}$ is the Lagrangian density. With respect to some path
in configuration space, $\vec{\Phi}(\vec{x}, t)$, the classical action becomes 
\begin{equation}
  S\left[\vec{\Phi}(\vec{x}, t) \right] = \int{dt\,L} 
  = \int{d^4x\,\mathcal{L}(\phi_k(\vec{x}, t),\partial_{\mu}\phi_k(\vec{x}, t), \vec{x}, t)}
\end{equation}
Using Hamilton's principle $\left( \frac{\delta S}{\delta \vec{\Phi} } = 0 \right)$ we obtain
the Euler-Lagrange equations 
\begin{equation}
  \frac{\partial \mathcal{L}}{\partial \phi_k}
  = \partial_\mu \left[ \frac{\partial \mathcal{L}}{\partial(\partial_\mu{\phi_k}) } \right]
\end{equation}
In classical mechanics we define the hamiltonian, $H(p_i(t), q_i(t), t)$, via the
Legendre transform (IS THIS ACTUALLY THE DEFINITION OF H? ASK INAKI)
\begin{equation}
  H = \sum_{i}{p_i(t)\dot{q}_i(t) - L}
\end{equation}
to generalise this to field theory we introduce the momentum field conjugate to 
$\phi_k(\vec{x}, t)$
\begin{equation}
  \pi_k(\vec{x}, t) \coloneq \frac{\partial \mathcal{L} }{\partial \dot{\phi}_k} \\
\end{equation}
\begin{equation}
  p_i(t) \coloneq \frac{\partial L}{\partial \dot{q}_i} \xrightarrow{i \rightarrow \vec{x}}
  \frac{\partial}{\partial \dot{\phi}_k} \left[ \int{d^3\vec{x}\,\mathcal{L}} \right]
  = \int{d^3\vec{x}\,\pi_k(\vec{x}, t)}
\end{equation}
Thus the classical Hamiltonian is generalised to
\begin{equation}
  H(t) = \int{d^3\vec{x}\,\mathcal{H}(\pi_k(\vec{x}, t), \phi_k(\vec{x}, t), \vec{x}, t) }
\end{equation}
\begin{equation}
  \mathcal{H} = \sum_{k}{\pi_k(\vec{x}, t)\dot{\phi}_k(\vec{x}, t) - \mathcal{L}}
\end{equation}
where $\mathcal{H}$ is the Hamiltonian density, and Hamilton's equations become
the Hamiltonian field equations
\begin{equation}
  \dot{\phi}_k = \frac{\partial \mathcal{H}}{\partial \pi_k}
  - \partial_\mu \left[ \frac{\partial \mathcal{H}}{\partial (\partial_\mu \pi_k)} \right],\,
  \dot{\pi}_k = \partial_\mu \left[ \frac{\partial \mathcal{H}}{\partial (\partial_\mu \phi_k)} \right]
  - \frac{\partial \mathcal{H}}{\partial \phi_k}
\end{equation}

[I tried not to use this notation but the rest will be unreadable if I don't, so I hope this
does not contradict following variational deriv notation, also use this notation earlier]
\begin{equation}
  \frac{\delta \mathcal{F}}{\delta g} \coloneq \frac{\partial \mathcal{F}}{\partial g}
  - \partial_\mu \left[ \frac{\partial \mathcal{F}}{\partial (\partial_\mu g)} \right],
\end{equation}

To begin second quantisation we will also require the notion of a field theoretic poisson bracket.
Given two functionals of the dynamical fields F and G given by 
\begin{equation}
  F = \int{d^3\vec{x}\, \mathcal{F}(\pi_k, \phi_k, \vec{x}, t)},\,
  G = \int{d^3\vec{x}\, \mathcal{G}(\pi_k, \phi_k, \vec{x}, t)}
\end{equation}
We define the poisson bracket
\begin{equation}
  {\{F, G \}}_{\phi,\pi} = \int{d^3\vec{x}\,
  \sum_{k}{\left[ \frac{\delta \mathcal{F}}{\delta \phi_k}\frac{\delta \mathcal{G}}{\delta \pi_k} 
- \frac{\delta \mathcal{G}}{\delta \phi_k}\frac{\delta \mathcal{F}}{\delta \pi_k} \right]}  }
\end{equation}
Note that $K({x}) = \int{d{y}\,[K({y}) \delta(x - y)]}$


\subsection{Noether's Theorem}

In classical mechanics Noether's theorem shows the correspondence between global symmetries of 
the lagrangian and conserved quantities called Noether charges. This generalises to global symmetries
of the lagrangian density corresponding to conserved Noether currents in classical field theory.
Consider an infinitesimal field transformation, $\vartheta(\epsilon)$, such that
\begin{equation}
  \phi_k \mapsto \phi_k + \epsilon\vartheta_k(\phi_k)
\end{equation}
where each $\vartheta_k$ may be a function of an arbitrary number of fields $\phi_k$ but
is independent of spacetime, and we suppress high order terms in $\epsilon$.
Such a transformation is called a global symmetry if its effect on the Lagrangian density is
\begin{equation}
  \mathcal{L} \mapsto \mathcal{L} + \epsilon\partial_\mu\Lambda^\mu(\phi_k, \vec{x}, t)
\end{equation}
This change in the Lagrangian density leaves the Euler-Lagrange equations invariant (add proof L8r).
Noether's theorem for fields states that a transformation $\vartheta_k(\epsilon)$ is a symmetry
of the lagrangian if and only if 
\begin{equation}
  j^\mu \coloneq
  \sum_{k}{\vartheta_k \frac{\partial \mathcal{L}}{\partial (\partial_\mu \phi_k)}}
  - \Lambda^\mu, \, \partial_\mu j^\mu = 0 
\end{equation}
and the divergence free quantity $j^\mu$ is called the conserved Noether current (add proof L8r).
\subsection{Second Quantisation}
In classical mechanics, the canonical transformations are those which preserve the symplectic 
structure; poisson brackets of the dynamical variables ($p_i, q_i$) are invariant under such 
transformations. A system's position in phase space specifies its classical state.

However, in quantum mechanics all properties of a system are included in a quantum state, $\ket{\psi}$, 
inside of a Hilbert space upon which operators corresponding to observables act.
Dirac's famous canonical quantisation
rule (${\{A, B \} \rightarrow \frac{1}{i\hbar}[\hat{A}, \hat{B}]}$) allows us quantise 
the canonical structure of classical mechanics (although Groenewold's theorem shows us 
that such a rule cannot hold for all functions of the dynamical variables). 
To obtain a quantised field theory we consider the field theoretic poisson brackets
\begin{equation}
  \{ \phi_n(\vec{z}), \phi_m(\vec{w}) \} = 0 = \{ \pi_n(\vec{z}), \pi_m(\vec{w}) \}
\end{equation}
\begin{equation}
  \{\phi_n(\vec{z}), \pi_m(\vec{w}) \} = \delta_{nm}\delta^{(3)}(\vec{z} - \vec{w})
\end{equation}
Which are canonically quantised to the commutator and anti-commutator relations 
for bosonic and fermionic fields operators respectively (consequence of spin-statistics)
acting on a Fock space
\begin{equation}
  {[\hat{\phi}_n(\vec{z}), \hat{\phi}_m(\vec{w}) ]}_\pm = 0
  = {[\hat{\pi}_n(\vec{z}), \hat{\pi}_m(\vec{w}) ]}_\pm
\end{equation}
\begin{equation}
  {[ \hat{\phi}_n(\vec{z}), \hat{\pi}_m(\vec{w}) ]}_\pm
  = i\hbar\delta_{nm}\delta^{(3)}(\vec{z} - \vec{w})
\end{equation}

\section{QED Lagrangian}
Introduce dirac eqn as relativistic QM,
show U(1) global symm. and that gauging this symm leads to QED lagrangian.
Show we now wish to gauge the axial symm. because of the prev. success but 
at the quantum level there is an anomaly acting as an obstruction to gauging.

In quantum mechanics the Schrodinger equation is not Lorentz invariant and is thus
incompatible with special relativity. A naive generalisation of the relativistic
dispersion relation using the quantum mechanical
Hamiltonian and momentum operators for free particles leads to the Klein-Gordon
equation. However, the Klein-Gordon is a second order partial differential equation and
subsequently does not uniquely determine time-evolution of the wavefunction.
A more complicated treatment produces the Dirac equation
\begin{equation}
  (i \gamma^\mu\partial_\mu - m)\psi = 0
\end{equation}
Which corresponds to the Lagrangian density
\begin{equation}
  \mathcal{L} = \bar{\psi}(i \gamma^\mu\partial_\mu - m)\psi
\end{equation}
Where the adjoint spinor is defined as $\bar{\psi} \coloneq \psi^\dagger\gamma^0$.
This Lagrangian admits a global $\mathrm{U}(1)$ symmetry
\begin{equation}
  \psi \mapsto e^{i\vartheta}\psi, \quad \bar{\psi} \mapsto \bar{\psi}e^{-i\vartheta}  
\end{equation}
This symmetry corresponds to a conserved current via Noether's theorem for fields
\begin{equation}
  j^\mu = \bar{\psi}\gamma^\mu\psi
\end{equation}





***NOT SURE IF THIS WILL BE NEEDED

The fields in the Lagrangian density are relativistic, 
meaning under Lorentz tranformations
\[ x'^{\mu} = \Lambda^{\mu}_{\nu} x^{\nu} \]
they transform as 
\[ \phi_i'(x') = R_l^j \phi_i(x) \]

**what is i? Does it matter?

We use the Minkowski metric 
\[ \eta_{\mu \nu} = diag(-1,1,1,1) \]
in these notes.

***DOWN TO HERE 

Define a symmetry

State Noether's thm



\section{Path Integral \& Generating Functional}
Given a particle position $x$ at time $t$ the probability amplitude of
a measurement at time $t'$ observing a position $x'$ is

\begin{align*}
  \begin{split}
  M(x', t'; x, t) &= \prescript{}{H}{\braket{x',t'|x, t}_{H}}
  \end{split} \\
  \begin{split}
  &= \braket{x'|\exp{\left[\frac{-i}{\hbar}\hat{H}(t-t')\right]}|x}
  \end{split}
\end{align*}\label{eq:1}
now we note that if the states are normalised
\begin{equation}
  \int{dy |y,t\rangle{} \langle{y,t|} } = \mathbb{I}
\end{equation}
we can write 
  \begin{equation}
    M(x', t'; x, t) = \sum_{n}{\psi_{n}(x')\psi_{n}^*(x)\exp{\left[\frac{-i}{\hbar}\hat{H}(t-t')\right]}}
  \end{equation}
which acts as our propagator
\begin{equation}
 \psi(x', t') = \int{dx \ M(x', t'; x, t) \psi(x, t)}
 \partial^{x}_{\mu}
\end{equation}


\section{Scalar Fields \& Green's Functions}
\section{Wess-Zumino Consistency \& Ward Identity}
\section{Stora-Zumino Descent}
\section{Dai-Freed \& Index Theorems}

\newpage
\bibliographystyle{unsrt}
\bibliography{reportNotes}
\addcontentsline{toc}{section}{References}

\end{document}
